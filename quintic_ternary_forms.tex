\section{Quintic Ternary Forms}

In this section, we will use the same notations used in Section \ref{general_sec}.
We ask whether we can follow the construction in Section \ref{general_sec} with $d=5$ to obtain a sum of 5 squares 
$\sum_{i=1}^5 p_i^2$ such that we have the finite maximum amount of real zeros for $Z(p_1,\dots, p_5)$ i.e. 
if $\# Z_{\R}(p_1,\dots, p_5) = 14$ is possible.

We will provide a strategy of obtaining such an example. 

\begin{enumerate}
\item First we choose initial forms in the $4\times 4$ matrix $B$ to be such that $\det B$ represents a curve with 
sufficiently many real inflections. Intuitively, this should provide us with a curve that is optimaly designed to 
intersect other curves at maximum number of real points. A good idea would be to choose $B$ to be the degree $4$ Taylor 
series approximation of a sinusoidal curve (because we have degree $4$, we could expand cosine at $0$ and obtain a 
degree $4$ polynomial estimate of the graph of the cosine function). Recall that that $q=\det B$.
\item Next we endow the linear forms $a_{i,1}$ for $i=1,\dots, 4$ with coefficients of $x,y,z$ that are paremetrized 
by parameters that we will want to solve. There are $15$ choices of parameters, we denote them as 
$$\bm b := b_1,\dots, b_{15}$$
\item We then obtain $p_2(\bm b;x,y,z),\dots, p_5(\bm b;x,y,z)$ paremetrized by $\bm b$
\item Furthermore, we can compute the first row of $\adj(B^\top)$ which we denote
$$\bm v := (\tilde b_{2,1},\dots, \tilde b_{5,1})$$
This is a vector of cubic forms in $x,y,z$
\item We also know the vector $\bm c$ in the right-hand side of equation \eqref{Bac_eqn} that has terms which are
quadratic in the $\bm b$.
\item We choose generic $14$ real points $P_1,\dots, P_{14} \in \P^2(\bar \Q\cap\R) \cap Z(q)$.
\item We would like to choose the parameters $b_1,\dots, b_{15}$ such that $p_2$ vanishes at $P_1,\dots, P_{14}$. 
This is the same as requiring that the inner product below vanishes
$$\spn{\bm v, \bm c}(\bm b;P_i) = 0$$
Thus we get $14$ quadratic equations in $\bm b$. 
\item Our objective is to find rational solutions (i.e.\ over $\Q$) to the $\bm b$. But this is not guaranteed. What 
is highly likely is to find a real algebraic solution $\bm b$. 
\item We then approximate $\bm b$ to rational parameters and use this. All the solutions to $Z(p,q)$ whould remain real
because $P_1,\dots, P_{14}$ was initially chosen generically (thus they should not be points were $\bm v$ vanishes) 
and real and so the new rational approximate to $\bm b$ will give us a slight perturbation of $P_1,\dots,P_{14}$ as 
points of $Z(p,q)$ which remain real.
\end{enumerate}
