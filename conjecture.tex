\section{conjecture}

Let $K:= \Q(\alpha)$ for some algebraic number $\alpha$ of degree $d=[K:\Q]>1$. 
Consider an element $P\in\P^n(K)$, then we say that $P$ is $\Q$-rational if there is a representative
$P=(p_0:p_1:\dots:p_n)$ with $p_i \in \Q$. We say that $P$ is a {\it general point over $K$} if for a 
representative $(p_0:p_1:\dots :p_n)$ of $P$, the coordinates $p_i$ generate a $\min(d,n)$-dimensional $\Q$ vector 
subspace of $K$ (thus all representatives of $P$ generate such a subspace). Similarly we define a {\it general 
hyperplane over $K$} by identifying points in $(\P^n)^*$ with points in $\P^n$. Thus, if $\beta$ is an algebraic 
number of degree $d>n$. Then immediately from the definition, a general hyperplane in $\P(\Q(\beta))^n$ will have no 
$\Q$-rational point. More generally we have the following Lemma

\begin{lemma}\label{l-qrational}
If $\beta$ is an algebraic number of degree $d$, then a general linear space of 
dimension less than $d-1$ in a projective space $\P^n(\Q(\beta))$ with $n\ge d-1$ has no $\Q$-rational point.
\end{lemma}

Now consider our construction of $d$-adic ternary forms. By a {\it general construction with maximum number of real 
points} we mean one such that 
\begin{itemize}
\item $Z(p_1,\dots, p_d)$ has maximum number of real points i.e. $M:=\binom{d+1}2-1$ different real points which we 
denote $P_1,\dots, P_M$.
\item The points $P_i$ are general over $K:=\Q(\beta)$ where $\beta$ is a real algebraic number of degree $M$.
\item The points $P_i$ are in sufficiently general position in $\P^2(K)$ (this will be made precise).
\end{itemize}

We want to prove the following claims 
\begin{itemize}
\item For any $d\ge 3$ such a construction is possible 
\item For such a construction, the sum of squares of these $d$-adic ternary forms is not $\Q$-sos.
\end{itemize}

\begin{none}
We start with the moduli space of planar curves of degree $d$, this is the projective space of dimension 
$$N := \binom{d+2}2 -1 $$
%see Beltrammeti \S p172, p'188, Remark 6.3.2
This is also a regular linear system, so that we can choose real points $P_1,\dots, P_M$ defined over $\Q$ (i.e.\ the 
set of these points is the zero set of a system of polynomial equations over $\Q$) general enough such that: 
$P_i$ is a $\Q(\beta)$-rational for some real algebraic number $\beta$ of degree $M$ and general over $\Q(\beta)$;
each $P_i$ imposes an independent linear condition to the linear system (the third condition above). So that with the 
base points $P_1,\dots, P_M$, we have a 
new linear system of dimension $N-M=d+1$. By Lemma \ref{l-qrational}, there are 
no non-trivial form over $\Q$ of degree $d$ that has all these $P_i$ as zero.
\end{none}

%jcapco now
From the above argument, we see that we can relax the condition for $M$. We need not require $M$ to be the 
maximum number of real points. It suffices to satisfy the two last 
conditions and instead of $M=\binom{d+1}2-1$ for the first condition, we simply require that $M > N-M$ i.e. 
% M>N-M \ge d+1
\begin{prop}
A general construction of $2d$-adic ternary form over $\Q$ ($d\in 2\N+1$) such that 
\begin{itemize}
\item $Z(p_1,\dots, p_d)$ has at least $M$ real points with 
$$ M > \frac{d(d+3)}4$$
\item The points $P_i$ are general over $K:=\Q(\beta)$ where $\beta$ is a real algebraic number of degree $M$.
\item The points $P_i$ are in sufficiently general position in $\P^2(K)$.
\end{itemize}
is not $\Q$-sos.
\end{prop}

\noindent
Thus, we see that for $d=3$ we need $M$ to be the maximum number of real points ($5$ points) and for $d=5$ it 
suffices to have $14\ge M \ge 11$ real points. %($M=14$ is the maxumum number of real points for $d=5$).
